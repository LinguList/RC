%basic cover letter template
\documentclass{letter}

\oddsidemargin = .2in
\evensidemargin = .2in
\textwidth = 6.3in
\topmargin = -.5in
\textheight = 9in

\newcommand {\qed}{\mbox{$\Box$}}
\renewcommand {\iff}{\Longleftrightarrow}
\newcommand {\R}{\mathbb{R}}
\newcommand {\N}{\mathbb{N}}
\newcommand {\Q}{\mathbb{Q}}
\newcommand {\Z}{\mathbb{Z}}

\newcommand {\sub}{\mbox{SB}}

\address{Molly L. Lewis\\
Department of Psychology\\
Stanford University\\
Jordan Hall\\
450 Serra Mall\\
Stanford, CA  94305-2130\\
\\
mll@stanford.edu}
\begin{document}

\begin{letter}

% This letter should contain two (100-word or shorter) summaries: a c oncise paragraph to the editor indicating the scientific grounds why the paper should be considered for a topical, interdisciplinary journal rather than for a single-discipline or archival journal; and a separate, 100-word summary of the paper's appeal to a popular (non-scientific) audience.

% The cover letter should state clearly what is included as the submission, including number of words in the text and number of display items (figures, tables, boxes) in the print version of the paper; number of additional words in the text (full Methods and Extended Data legends) and number of Extended Data figures and tables for the online-only version; any Supplementary Information (specifying number of items and format); number of supporting manuscripts.

\opening{Editorial Board\\ 
Cognition\\
\\
Dear Editors,}

Please accept the manuscript ``The length of words reflects their conceptual complexity," which I am submitting with my coauthor Michael C. Frank. The manuscript reports original data revealing a new regularity in language: longer words tend to refer to more complex meanings. In a series of ten studies, we find this regularity both in speakers' inferences about novel words, as well as in the structure of natural language. Our data suggest that pragmatic pressures may constrain the arbitrary character of the lexicon.

This manuscript is not currently submitted elsewhere and will not be submitted elsewhere prior to editorial decision.

APA ethical standards on the collection of data from human participants were followed. My coauthor and I do not have any interests that might be interpreted as influencing the research.

I will be serving as the corresponding author for this manuscript. My coauthor and I have agreed to the byline order and to submission of the manuscript in this form. I have assumed responsibility for keeping my co-author informed of our progress through the editorial review process, the content of the reviews, and any revisions made. I understand that, if accepted for publication, a certification of authorship form will be required that all authors will sign.

%A fundamental assumption about the nature of human languages is that their words are arbitrary. The word ``horse,'' for example, doesn't look or sound any more or less like an actual horse than any other word does. Yet our work reveals a striking regularity within the lexicons of natural languages: longer words have meanings that are more conceptually complex. We find this bias both in experimental work with novel words and in corpus work across 80 natural languages. In addition, this bias is not accounted for by other linguistic characteristics of words (like their frequency or predictability). Our findings suggest a universal tendency across languages for word forms to be related to the cognitive complexity of their meaning.

% the paper's appeal to a popular (non-scientific) audience.
%This work will also have appeal to a non-scientific audience. Our research provides insight into a simple, but puzzling question about language: Why are some words easier to say than others? Our findings suggest that one answer to this question may be the role of communication in language. In everyday communication, we tend to use  longer sentences to refer to more complex events. For example, a speaker might say ``Lee got the car to stop," as opposed to the shorter ``Lee stopped the car," in order to suggest that something unusual happened (perhaps the car brakes failed). Our work reveals that languages also show this regularity, using longer labels to refer to more complex meanings.

Sincerely,\\
\\
\\
Molly L. Lewis\\
Department of Psychology\\
Stanford University\\

\end{letter}

\end{document}